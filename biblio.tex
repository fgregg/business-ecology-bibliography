\documentclass{article}

\usepackage{natbib,bibentry}
\bibliographystyle{apalike}

\begin{document}
\nobibliography{classics,factorial,interaction,tnc,site_selection,org_ecology,cluster}

The quality and character of urban districts are determined, to a
large extent, by the ecological relations among people and
organizations. Who lives in a district affects who else will move in
and what businesses will open and persist. Likewise, the retail mix
affects both what other businesses will appear and what type of people
will be attracted to the area.

Historically, competition has been the most important type of
ecological interaction for shaping patterns of land use. Overall
competition for minimal travel time produced a gradient of prices, and
competition between industrial, commercial, and residential uses
determined the overall character of large swaths of a cities. Within
residential areas, competition between ethnicities produced
residential segregation. 

Agglomeration economies -- exception special districts (it's becoming rule)

Within residential areas, the density and types of people represented
a 


type of ecological interaction has been
competition. Competition for all 

Competition between  determined the overall character of large swaths of
cities. Within residential areas, competition between of large determined the overall nature of an area. But within
areas, the main ecological interactions has been among people within
residential areas and among businesses within business
districts. While the density and types of people living an area
determined the number and types of retail, the interactions 


The rise of city dwelling as lifestyle choice has increased the
importance of the ecological interactions between people and
businesses. Urbanites increasingly base their residential location on the
presence of amenity businesses that server their particular
interests. 

The rise of knowledge work and it's networked modes of production
make the patterns of


ecological interaction between people and
businesses are 


The presence of certain types of people within a district
change the likelihood that other types will move in. The presence of
certain types of 

I'm interested in how different land used come to be arranged across
urban space, particularly in it's emergent moment. How does this
presence of a certain type of land use make other types of proximate
land use more or less likely. 

The cl

How do different types of land uses become arranged across urban
space? The classic answer, has been 

* zoning, politics, 

\section*{Classics}
\begin{enumerate}
\item \bibentry{park_city:_1984}
\item \bibentry{palmer_primary_1932}
\item \bibentry{mckenzie_pattern_1934}
\item \bibentry{cressey_population_1938}
\item \bibentry{gettys_human_1940}
\item \bibentry{firey_sentiment_1945}
\item \bibentry{harris_nature_1945}
\item \bibentry{hatt_relation_1945}
\item \bibentry{hatt_concept_1946}
\item \bibentry{firey_land_1947}
\item \bibentry{clark_urban_1951}
\item \bibentry{pappenfort_ecological_1959}
\item \bibentry{alonso_theory_1960}
\item \bibentry{hoyt_one_1970}
\item \bibentry{hoyt_structure_1972}
\item \bibentry{bailey_sociocultural_1972}
\item \bibentry{alihan_social_1975}
\item \bibentry{theodorson_urban_1982}
\item \bibentry{judd_city_2011}
\end{enumerate}

\section*{Factorial}
\begin{enumerate}
\item \bibentry{hawley_human_1950}
\item \bibentry{duncan_residential_1955}
\item \bibentry{shevky_social_1955}
\item \bibentry{bose_calcutta:_1964}
\item \bibentry{abu-lughod_cairo:_1971}
\item \bibentry{timms_urban_1975}
\item \bibentry{berry_contemporary_1977}
\item \bibentry{janson_factorial_1980}
\item \bibentry{hawley_urban_1981}
\item \bibentry{micklin_sociological_1984}
\item \bibentry{bogue_structure_1988}
\item \bibentry{fischer_subcultural_1995}
\item \bibentry{micklin_continuities_1998}
\item \bibentry{sampson_assessing_2002}
\end{enumerate}

\section*{Interactional}
\begin{enumerate}
\item \bibentry{schelling_dynamic_1971}
\item \bibentry{grannis_importance_1998}
\item \bibentry{janssen_ecology_2006}
\item \bibentry{luhmann_autopoiesis_2008}
\item \bibentry{mcdonald_urban_2010}
\item \bibentry{brueckner_lectures_2011}
\item \bibentry{heppenstall_agent-based_2012}
\end{enumerate}

\section*{Organizational Ecology}
\begin{enumerate}
\item \bibentry{hannan_population_1977}
\item \bibentry{freeman_niche_1983}
\item \bibentry{mcpherson_ecology_1983}
\item \bibentry{carroll_organizational_1984}
\item \bibentry{popielarz_edge_1995}
\item \bibentry{podolny_networks_1996}
\item \bibentry{audretsch_r&d_1996}
\item \bibentry{anselin_local_1997}
\end{enumerate}

\subsection*{Cultural Niches}

\subsection*{Local Market Analysis}
\item \bibentry{schuetz_rail_2014}
\item \bibentry{weiler_understanding_2003}
\item \bibentry{arentze_multipurpose_2005}
\item \bibentry{wendt_food_2008}
\item \bibentry{smith_assessment_2003}
\item \bibentry{porter_location_2000}
\item \bibentry{schuetz_are_2012}
\item \bibentry{bellinger_poverty_2011}

\section{Clusters}
\item \bibentry{audretsch_r&d_1996}
\item \bibentry{anselin_local_1997}
\item \bibentry{sassen_global_2013}
\item \bibentry{porter_competitive_1998}
\item \bibentry{rietveld_transport_2004}
\item \bibentry{david_emanuel_andersson_spatial_2012}
\item \bibentry{storper_buzz:_2003}
\item \bibentry{clark_scenes_????}
\item \bibentry{florida_rise_2014}
\item \bibentry{dear_chicago_2001}
\item \bibentry{glaeser_consumer_2000}

porter, saskia, clark

Insight that orgs can colocate because they need to be near one
another

\section{Site Selection}


\begin{enumerate}
\item \bibentry{sevtsuk_location_2014}
\end{enumerate}

\section*{Unclassified}
\begin{enumerate}
\item \bibentry{knudsen_walk_2013}
\item \bibentry{lindstrom_sense_1997}
\item \bibentry{beveridge_commonalities_2011}
\item \bibentry{geertz_agricultural_1969}
\end{enumerate}


\end{document}


%% Local Variables:
%% zotero-collection: #("239" 0 3 (name "Ecology/Classic Urban Ecology and It's Critics"))
%% zotelo-auto-update: t
%% End:
